%%%%%%%%%%%%%%%%%%%%%%%%%%%%%%%%%%%%%%%%%%%%%%%%%%%%%%%
%																					%
%	In dieser Datei werden alle Packages eingebunden, 	%
% welche f�r das Dokument n�tig sind. Desweiteren 		%
% werden die Dokumentinformationen gesetzt.						%
%																											%
%%%%%%%%%%%%%%%%%%%%%%%%%%%%%%%%%%%%%%%%%%%%%%%%%%%%%%%
%

%
\documentclass[pdftex, 		a4paper, 		% DIN A4 verwenden
							titlepage,	    % separate Titelseite
							%draft,			% Draft-Version, keine Bilder im pdf!
							emulatestandardclasses,
							final,			% Final-Version
							oneside,		% zweiseitiger Druck %X
							11pt,			% Schriftgr��e 12pt
							%BCOR20mm,		    			    %X
							DIV=calc,		
							headsepline,	
							%tocbasic,
							%openany,
							%pointlessnumbers
							]{article}	%	KOMAScript scrbook-Dokumentklasse							
%%%%%%%%%%%%%%%%%%%%%%%%%%%%%%%%%%%%%%%%%%%%%%%%%%%%%%%%
%	Einbinden der Pakete 
%%%%%%%%%%%%%%%%%%%%%%%%%%%%%%%%%%%%%%%%%%%%%%%%%%%%%%%%
\usepackage[a4paper, left=2.5cm, right=2.5cm, top=3cm, bottom=3cm, bindingoffset=0mm]{geometry}
%bindingoffset=15mm]

\usepackage[ngerman,english]{babel}
\usepackage{multirow}
\usepackage{amsmath}
\usepackage{amssymb}
\usepackage[pdftex]{graphicx}
\usepackage{chngcntr}
\counterwithin{table}{section}
\counterwithin{figure}{section} %X
\counterwithin{equation}{section} %X
\usepackage{booktabs}
% PDF Dateien einbinden
\usepackage{pdfpages}
\usepackage{textcomp} 
\usepackage{upgreek}%kein Kursiv mehr
\pdfminorversion=6
\pdfcompresslevel=9

% Abstand zu Section...
%\RedeclareSectionCommand[beforeskip=-1.5\baselineskip,afterskip=.5\baselineskip]{section}
%\RedeclareSectionCommand[beforeskip=-.75\baselineskip,afterskip=.5\baselineskip]{subsection}
%\RedeclareSectionCommand[beforeskip=-.5\baselineskip,afterskip=.25\baselineskip]{subsubsection}
%\RedeclareSectionCommand[beforeskip=.5\baselineskip,afterskip=-1em]{paragraph}

\usepackage{textcomp} %griechische Alphabet nicht kursiv

%Einige Pakete haben Probleme mit dem Komaskript.
\usepackage{scrhack} 

\usepackage{xcolor}
\definecolor{urlLinkColor}{rgb}{0,0,0.5}
\definecolor{LinkColor}{rgb}{0,0,0}

\usepackage{abstract} %Abstrakt

\usepackage{csquotes}
\usepackage[backend=biber,style= apa, sortcites = true]{biblatex}

\newcommand\ifintextcite{\ifdefstring{\blx@delimcontext}{textcite}} %parencite und textcite unterschiedlich
\DefineBibliographyStrings{german}{
	andothers = {\ifintextcite{and others}{et\addabbrvspace al.}},
%	nodate = {o.D.},
%	backrefpage  = {\lowercase :}
%	pages = {:}
}
\setlength{\bibitemsep}{12pt}
\addbibresource{source/bib/references.bib}
	
\usepackage[latin1]{inputenc} % Umlaute %  
\usepackage[scaled]{berasans}				% Schriftfamilie
\renewcommand{\rmdefault}{phv}
\renewcommand{\sfdefault}{phv}

% Grafikpaket
\usepackage{makeidx}   				% Paket zur Erzeugung eines Index
\usepackage[normalem]{ulem}   % bietet Unterstreichungsvarianten
%\usepackage{picins} 					% Bilder im Absatz platzieren
\usepackage[T1]{fontenc}			% Erweiterten Zeichensatz aktivieren
\usepackage{multido}					% erm�glicht Schleifenartiges wiederholen von Befehlen
\usepackage{mdwlist}					% erm�glicht das Setzen des Z�hlers bei Aufz�hlungspunkten
\usepackage{paralist}					% Paket f�r Aufz�hlungen, erweitert Enumerate-Paket
\usepackage{longtable}				% mehrseitige Tabellen
%\usepackage{tocloft}
\parindent0pt           			% verzichte auf Einr�cken der ersten Zeile
\parskip2ex            	% Abstand zwischen den Abs�tzen

\usepackage{floatflt,epsfig} %text neben Bild

%\usepackage{setspace}					% Paket zum Einstellen des Zeilenabstands
%\doublespacing								% doppelter Zeilenabstand
\linespread{1.5}


\usepackage{color}
\definecolor{hcu-blau}{RGB}{54, 141, 207}

% Farbeinstellungen f�r die Links im PDF Dokument.
%
\makeindex

%-----------Paket f�r absolute Positionierung von Grafiken------------------
\usepackage[absolute]{textpos}
\setlength{\TPHorizModule}{1mm}
\setlength{\TPVertModule}{\TPHorizModule}

%-----------Aufz�hlungen und Einstellungen f�r Sourcecode-------------------
%\usepackage[savemem]{listings} %Bei wenig Arbeitsspeicher dies Option [savemem] aktivieren.
\usepackage{listings}
\lstloadlanguages{TeX,XML, Java} % TeX sprache laden, notwendig wegen option 'savemem'
\lstset{%
	language=[LaTeX]TeX,     % Sprache des Quellcodes ist TeX
	numbers=left,            % Zelennummern links
	stepnumber=1,            % Jede Zeile nummerieren.
	numbersep=5pt,           % 5pt Abstand zum Quellcode
	numberstyle=\tiny,       % Zeichengr�sse 'tiny' f�r die Nummern.
	breaklines=true,         % Zeilen umbrechen wenn notwendig.
	breakautoindent=true,    % Nach dem Zeilenumbruch Zeile einr�cken.
	postbreak=\space,        % Bei Leerzeichen umbrechen.
	tabsize=2,               % Tabulatorgr�sse 2
	basicstyle=\ttfamily\footnotesize, % Nichtproportionale Schrift, klein f�r den Quellcode
	showspaces=false,        % Leerzeichen nicht anzeigen.
	showstringspaces=false,  % Leerzeichen auch in Strings ('') nicht anzeigen.
	extendedchars=true,      % Alle Zeichen vom Latin1 Zeichensatz anzeigen.
	backgroundcolor=\color{ListingBackground}} % Hintergrundfarbe des Quellcodes setzen.


\lstset{
  basicstyle=\small\ttfamily,
  columns=fullflexible,
  showstringspaces=false,
  %commentstyle=\color{gray}\upshape
}
%neue Lang definieren, als Bsp.
\lstdefinelanguage{XML-changed}
{
  basicstyle=\footnotesize\ttfamily\bfseries,
  morestring=[b]",
  morestring=[s]{>}{<},
  morecomment=[s]{<?}{?>},
  stringstyle=\color{black},
  identifierstyle=\color{darkblue},
  keywordstyle=\color{cyan},
  morekeywords={xmlns,version,type}% list your attributes here
}

%-----------Caption Package-------------------
\usepackage{caption}
\usepackage{subcaption}
\DeclareCaptionFont{hcu-blau}{\color{hcu-blau}}

\usepackage{silence}
\WarningFilter{scrbook}{Usage of package `fancyhdr'}

\usepackage[labelfont = {bf,it}, font={footnotesize,it}, labelsep = space]{caption} %Beschirftung Tabelle/Abbildung

%\usepackage[labelfont={hcu-blau,{bf}}]{caption}
\renewcommand{\arraystretch}{1.0}   %tabellenabstand

\usepackage[labelfont = {bf,it}, font={footnotesize,it}, labelsep = space]{subcaption}

%-----------Header+Footer---------------------------------------------------
%\usepackage{fancyhdr}					
%\pagestyle{fancy}
%\fancyhf{}							
%\lhead{\leftmark}
%\rhead{\DeinName}
%\rfoot{\thepage} 
%\fancyfoot[EL]{\thepage}   %Seitenzahl abwechselt
%\fancyfoot[OR]{\thepage}   %Seitenzahl abwechselt
%\renewcommand{\headrulewidth}{0.2pt} % Kopflinie
%\renewcommand{\sectionmark}[1]{\markleft{\arabic{chapter}.\arabic{section}\ #1}}

%--------------------------------------------------------------------------------

\usepackage{scrlayer-scrpage}
\pagestyle{scrheadings}
\clearpairofpagestyles

\ohead{\pagemark}
%\chead{Kopfzeile Mitte}
\ihead{\leftmark}
%\ifoot{Fu�zeile innen}
%\cfoot{Fu�zeile Mitte}
%\ofoot{\pagemark}

%\renewcommand*\chapterpagestyle{scrheadings}
\renewcommand{\sectionmark}[1]{\markleft{\arabic{section}\ #1}}
%--------------------------------------------------------------------------------

% -------F�r ToDo-Notes--------------------------------------------------------------------
\usepackage[color=red, shadow]{todonotes} % ", disable" deaktiviert ToDo-Notes
%Vereinfachtes "Inline-Todo"
\newcommand{\td}[1]{{\todo[inline]{#1}}}
\newcommand{\tdu}[1]{{\todo[inline, color=green!40]{#1}}}

%--------F�r Links-------------------------------------------------------------------------
%--------HyperRef konfigurieren-------------------------------------------------------------------------

\usepackage[
	pdftitle={\Titelpdf},
	pdfauthor={\DeinName},
	pdfsubject={\TitelArbeit},
	pdfcreator={MiKTeX, LaTeX with hyperref and KOMA-Script auf Basis der Vorlage von seiler.it},
	pdfkeywords={Bachelorarbeit, Hamburg, HafenCity Universit�t},%weitere Keywords hier einf�gen
	pdfpagemode=UseOutlines,%                                  
	pdfdisplaydoctitle=true,%                                  
	pdflang=de%                                              
]{hyperref}

\hypersetup{%
	colorlinks=true,%        Aktivieren von farbigen Links im Dokument (keine Rahmen)
	linkcolor=LinkColor,%    Farbe festlegen.
	citecolor=LinkColor,%    Farbe festlegen.
	filecolor=LinkColor,%    Farbe festlegen.
	menucolor=LinkColor,%    Farbe festlegen.
	urlcolor=LinkColor,%     Farbe von URL's im Dokument.
	bookmarksnumbered=true%  �berschriftsnummerierung im PDF Inhalt anzeigen.
}
