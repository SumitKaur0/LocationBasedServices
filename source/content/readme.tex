\ihead{ANHANG}
\section*{Anhang}
{\Large Erkl�rung des Python Codes}
\medskip\\
Eingangsdaten
EightGroundTruth.csv --> Ground Truth
EightStepTimes.csv --> Schrittzeiten
nfg11.csv --> 5G Koordinaten hohe Frequenz hohe Genauigkeit
nfg53.csv --> 5G Koordinaten niedrige Frequenz niedrige Genauigkeit
nsh.csv --> Schrittl�ngen
nsl.csv --> Schrittrichtungen
Kalman Filter
MyKalmanFilterGNSSODO.py
main()
Ruft data\_import() auf um die Daten einzulesen.
Es werden die Nullwerte festgelegt.
Ruft calc\_delta() auf um Delta Werte zu berechnen aus Schrittl�ngen und Schrittrichtungen.
Ruft initial\_covariance() auf um die Kovarianzmatrix der Parameter zu beginn zu definieren.
Ruft Kalman\_Filter() auf den Kalman Filter durchlaufen zu lassen.
Ruft die Plot Funktion auf.
data\_import()
%Importiert Daten aus den .csv Dateien. �nderungen an den importierten Daten m�ssen hier vorgenommen
%werden.
%Es werden die Daten zur�ckgegeben.
%calc_delta()
%Berechnet.
%Es werden die Delta Werte zur�ckgegeben.
%Kalman_Filter()
%L�uft iterativ durch die Beobachtungsdaten (for Schleife).
%Genauigkeit von Delta x und Delta y abh�ngig von Anzahl der Schritte (while Schleife).
%README.md 8/19/2022
%2 / 4
%Ruft prediction() auf.
%Ruft correction() auf.
%Es wird das Endergebnis zur�ckgegben.
%prediction()
%In dieser Funktion findet der Prediktionsschritt des Kalman Filters statt
%Zur�ckgegeben wird der pr�dizierte x-Vektor und die dazugeh�rige Kovarianzmatrix.
%correction()
%In dieser Funktion wird die Innovation durchgef�hrt, die Kalman Gain Matrix erstellt und das Update
%berechnet.
%Es wird der korrigierte x-Vektor zur�ckgegeben und die dazugeh�rige Kovarianzmatrix.
%plot()
%Funktion zum plotten der Ergebnisse
%Least Square
%ls5G.py
%main()
%L�dt Daten aus .csv Dateien ein.
%Genauigkeitsparameter s1 und s2 werden definiert.
%Funktion ls() wird aufgerufen
%ls()
%L�uft iterativ durch die Beobachtungen (for Schleife)
%5G Beobachtungen werden f�r jeden Schritt simuliert durch addieren der Schrittl�nge + Schrittrichtung auf
%die letzten 5G Koordinaten (while Schleife).
%Ruft in jedem Schritt die WLS() Funktion auf.
%Gibt die gesamte optimierte Trajektorie zur�ck.
%WLS()
%In dieser Funktion findet das berechnen der Trajektorie statt.
%WLS gibt die optimierte Trajektorie f�r den einzelnen Schritt wieder aus.
%plot()
%README.md 8/19/2022
%3 / 4
%Diese Funktion dient zum plotten der Ergebnisse
%Extended Kalman Filter
%MyExtendedKalmanFilterGNSSODO.py
%main()
%Ruft data_import() auf.
%Definiert Startwerte.
%Ruft calc_delta() auf.
%Ruft initial_covariance() auf um die Kovarianzmatrix der Parameter zu beginn zu definieren.
%Ruft Kalman_Filter().
%data_import()
%
%Importiert Daten aus den .csv Dateien. �nderungen an den importierten Daten m�ssen hier vorgenommen
%werden.
%Es werden die Daten zur�ckgegeben.
%calc_delta()
%Berechnet.
%Es werden die Delta Werte zur�ckgegeben.
%Kalman_Filter():
%L�uft iterativ durch die Beobachtungsdaten (for Schleife).
%Schritte und Richtungen bis zur n�chsten 5G Koordinate werden gesammelt (while Schleiife).
%prediction() wird aufgerufen.
%correction() wird aufgerufen.
%Endergebnis wird zur�ckgegeben.
%prediction()
%Transitionsmatrix wird ensprechend der Anzahl der eingehenden Beobachtungen gebildet (for Schleife)
%Zur�ckgegeben wird der pr�dizierte x-Vektor und die dazugeh�rige Kovarianzmatrix.
%correction()
%In dieser Funktion wird die Innovation durchgef�hrt, die Kalman Gain Matrix erstellt und das Update
%berechnet.
%README.md 8/19/2022
%4 / 4
%Funktion wird angepasst im Gegensatz zum diskreten Kalman Filter (siehe oben).
%Die Kovarianzmatrix der Beobachtungen wird dynamisch, je nach Anzahl der Schritte zum n�chsten 5G Punkt,
%angepasst (for Schleife).
%Es wird das der korrigierte x-Vektor zur�ckgegeben und die dazugeh�rige Kovarianzmatrix.