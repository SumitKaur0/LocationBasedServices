\section{Einleitung}

\section{Das dynamischer Kalman Filter}
Eicker:\\
Setzt sich zusammen aus einem Beobachtungsmodell und einem Bewegungsmodell zusammen. Bei dem Beobachtungsmodell handelt es sich um Beobachtungen und ihre Unsicherheiten. Bei dem Bewegungsmodell spricht wird auch von einer Pr�diktion (dynamisches Modell) und ihren Unsicherheiten gesprochen. Aus ihr erfolgt dann die Sch�tzung des Zustandes. 

\subsection{Diskreter Kalmanfilter (lineares Modell)}
Idee: Beobachtungen zu bestimmten (diskreten) Zeitpunkten mit einem Bewegungsmodell kombinieren. Das Bewegungsmodell pr�diziert dann den Zustandsvektor ausgehend von der Sch�tzung des vorherigen Schrittes.

Beobachtungsmodell:\\
$l_{i+1} = A_{i+1}x_{i+1} + e_{i+1}$\\
l = Beobachtungen/Messung\\
A = Designmatrix\\
e = Residuen (Beobachtungsrauschen) --> e = -v\\
Residuen sind normalverteilt, Erwartungswert = 0\\

Bewegungsmodell:\\
$x_{i+1} = T_ix_i + C_iw_1$\\
T = Transitionsmatrix (pr�diziert Bewegung von einem zum n�chsten Zeitpunkt)\\
w = St�rgr��e (Unsicherheit im Bewegungsmodell -> Rauschen)\\
C = St�rgr��enmatrix (Auswirkung dieser Unsicherheit auf die Pr�diktion des Zustandes)\\
St�rgr��e normalverteilt, Erwartungswert = 0 -> beeinflusst nur das stochastisches Modell aber nicht die eigentliche Pr�diktion\\

1. Pr�diktion: \\
Welche Trajektorie des Fahrzeugs sagt das Bewegungsmodell voraus?\\
$\bar{x}_{i+1} = T_i + \hat{x}_{i}$\\
$\sum(\bar{x}_{i+1}) = T_i\sum(\hat{x}_{i})T_i^T + C_i\sum(w_{i})C_i^T$\\

2. Innovation:\\
Welche Trajektorie des Fahrzeugs sagt das Bewegungsmodell voraus? Was behaupten die Beobachtungen? Wie sehr weichen die Beobachtungen von der Pr�diktion ab? => Innovation\\
$d_{i+1} = l_{+1} - A_{i+1}\bar{x}_{i+1}$\\
$\sum(d_{i+1}) = \sum(l_{i+1}) + A_{i+1}\sum(\bar{x}_{i+1})A_{i+1}^T$

3. Gain Matrix (K-Matrix):\\
Relative Gewichtung von Pr�diktion und Beobachtungen anhand der jeweiligen Genauigkeiten\\
$K_{i+1} = \sum(\bar{x}_{i+1})A_{i+1}^T\sum^{-1}(d_{i+1})$

4. Update: \\
Gewichtetes Mittel aus Pr�diktion und  Innovation\\
$\hat{x}_{i+1} = \bar{x}_{i+1} + K_{i+1}d_{i+1}$\\
$\sum(\hat{x}_{i+1}) = [I-K_{i+1}A_{i+1}]\sum(\bar{x}_{i+1})$

Geschichte des Kalman-Filters:\\
Erdunfen von Rudolf E. Kalman (Transcations of the ASME-Journal of Basic Engineering, 82 (Series D): 35-45.Copyright \copy 1960 by ASME)

Gut geeignet, um die Bahnen von Raketen zu berechnen (der Apollo Mondmission)

Dynamisches Modell: Trajektorie der Mondrakete
Beobachtungen: Space sextant, inertial navigator (Weltraumsextant, Tr�gheitsnavigator)

\subsection{Extended Kalmanfilter (nicht- lineares Modell)}
Weder Bewegungsmodell noch das Beobachtungsmodell ist linear, deshalb wird das extended Kalman Filter gebraucht.\\
$x_{i+1} = f^{i+1}_i (x_i, w_i)$\\
$l_{i+1} = a_{i+1}(x_{i+1}) + e_{i+1}$\\

Bei nicht-linearen Zusammenh�ngen werden die Matrizen A, T und C durch Linearisierung (partielle Ableitungen) der nicht-linearen Funktionen f und a bestimmt:

$T = f(x,w) abgeleitet nach x$\\
$C = f(x,c) abgeleitet nach w$\\
$A = a(x) abgeleitet nach x$\\

Inpur: \\
$x_{i+1} = f^{i+1}_i (x_i, w_i)$\\

\subsubsection{Weitere Quellen:}
https://www.cbcity.de/das-kalman-filter-einfach-erklaert-teil-2

\bibliography