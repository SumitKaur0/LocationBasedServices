\section{Introduction}
Storytelling in many ways are a big part of many subjects in school. Especially when it comes to a subject like geography, history or social studies, it is often essential to understand the temporal and spatial context of a story. This is when map based Storytelling can play to its strengths. Such Story Maps can visualize spatial data and temporal order at the same time while including the content of the story. Furthermore, visual storytelling offers hybridization in cartography through combining technology and multimedia. Story Maps encourage non-expert user groups to participate in map creation. They can use it to generate their own location-based stories \parencite{kerski2015}.

\bibliography